\chapter{Исследовательская часть}

\section{Постановка исследования}

Целью исследования является определение времени пересчета полигонов поверхности жидкости от количества полигонов.


\section{Технические характеристики}

Технические характеристики устройства, на котором выполнялось тестирование:
\begin{itemize}[label={---}]
	\item операционная система MacOS Ventura 13.5.2;
	\item 16 ГБ оперативной памяти;
    \item процессор 2,6 ГГц 6‑ядерный Intel Core i7~\cite{intel}.

\end{itemize}

Во время тестирования устройство было подключено к сети электропитания, нагружено приложениями окружения и самой системой тестирования.

\section{Время выполнения алгоритмов}

Алгоритмы тестировались при помощи функции $process\_time()$ из библиотеки $time$~\cite{pythonlangtime} языка $Python$~\cite{python}. 
Данная функция возвращает текущее значение системного процессорного времени в секундах.

Замеры времени для каждого количества полигонов проводились 20 раз. 
В качестве результата взято среднее время работы алгоритма. 

В таблице~\ref{tbl:time} приведены результаты замера времени (в микросекундах). 


\begin{table}[H]
\caption{\label{tbl:time}Результаты измерений времени пересчета полигонов поверхности жидкости}
\centering
\begin{tabular}{|r|r|}
\hline
Количество полигонов, штук & Время, микросекунды \\ \hline
8    &    213 \\ \hline
32   &    785 \\ \hline
72   &  1 753 \\ \hline
128  &  3 112 \\ \hline
200  &  4 843 \\ \hline
288  &  6 988 \\ \hline
392  &  9 506 \\ \hline
512  &  12 387 \\ \hline
648  &  15 489 \\ \hline
800  &  18 840 \\ \hline
968  &  22 818 \\ \hline
1 152 &  26 544 \\ \hline
1 352 &  30 861 \\ \hline
1 568 &  35 217 \\ \hline
\end{tabular}
\end{table}

По таблице \ref{tbl:time} был построен график на рисунке \ref{img:graph_time}

\img{90mm}{graph_time}{График зависимости измерений времени пересчета полигонов поверхности жидкости от количества полигонов}

\newpage

Как видно из графика, время пересчета полигонов поверхности жидкости линейно зависит от количества полигонов.

\section*{Вывод}

В данном разделе были исследованы результаты замеров времени пересчета полигонов поверхности жидкости от количества полигонов.

