\chapter{Аналитический раздел}

В данном разделе проводится анализ существующих алгоритмов построения изображений и выбор подходящих алгоритмов для решения задачи.

\section{Формализация объектов сцены}

Сцена состоит из следующих объектов:
\begin{itemize}[label*=---]
    \item поверхность жидкости -- трехмерная модель, представляющая собой полигональную сетку, состоящую из связанных между собой плоских треугольников;
    \item раковина -- трехмерная модель, которая содержит в себе поверхность жидкости;
    \item капля -- трехмерная модель, возбуждающая волны на поверхности жидкости;
    \item источник света -- вектор направления света;
    \item камера -- характеризуется своим положением и направлением просмотра.
\end{itemize}

\section{Анализ способов описания трехмерных моделей}

В компьютерной графике существуют три основных типа моделей для описания трехмерных объектов: каркасная, поверхностная и твердотельная модели. 
Они предоставляют различные способы представления объектов и позволяют 
достичь правильного отображения их формы и размеров на сцене.

\subsection{Каркасная модель}
Каркасная модель –- в трехмерной графике описывает совокупность вершин и ребер, которая показывает форму многогранного объекта. 
Это моделирование самого низкого уровня и имеет ряд серьезных ограничений, большинство из которых возникает из-за недостатка информации о гранях, которые заключены между линиями, и невозможности выделить внутреннюю и внешнюю область изображения твердого объемного тела. 
Однако каркасная модель требует меньше памяти и вполне пригодна для решения задач относящихся к простым. 
Основным недостатком каркасной модели является то, что модель не всегда однозначно передает информацию о форме объекта~\cite{MTM}.

\subsection{Поверхностная модель}
Поверхностная модель определяется в терминах точек, линий и поверхностей. 
При построении поверхностной модели предполагается, что технические объекты ограничены поверхностями, которые отделяют их от окружающей среды. 
Недостатком поверхностной модели является отсутствие информации о том, с какой стороны находится поверхности материала~\cite{MTM}.

\subsection{Твердотельная модель}
Твердотельная модель -- отличается от поверхностной модели поверхности тем, что она содержит информацию о том, каким образом материал расположен с той или иной стороны. 
Для достижения этого, в модели указывается направление внутренней нормали.

\subsection{Выбор способа описания модели}

Для решения поставленной задачи наилучшим образом подойдет поверхностная модель.

Поверхностная модель задается полигональной сеткой. 
Полигональная сетка характеризуется совокупностью вершин, ребер и граней, определяющих форму объекта в трехмерном пространстве.

Изначально для представления поверхности жидкости используется сеточная модель, состоящая из массива точек (вершин сетки), которые разбиваются на треугольники. 
Для вычисления координат точек поверхности жидкости на каждом шаге времени необходимо использовать волновое уравнение. 
Это позволяет быстро определить новые координаты точек сетки и обновить их значения.

\newpage 

\section{Анализ алгоритмов удаления невидимых поверхностей}

Удаление невидимых линий и поверхностей -- одна из самых сложных задач в графике. 
Алгоритмы удаления невидимых линий и поверхностей определяют, какие линии, поверхности или объемы видимы или невидимы для наблюдателя, находящегося в определенной точке пространства.

\subsection{Алгоритм Робертса}

Алгоритм Робертса работает в объектном пространстве только с выпуклыми телами. 
Если тело не является выпуклым, то его предварительно нужно разбить на выпуклые составляющие~\cite{ROB}.

Этот алгоритм выполняется в 4 этапа:
\begin{itemize}[label*=---]
    \item Подготовка исходных данных – составление матрицы тела для каждого тела сцены;
    \item Удаление ребер, экранируемых самим телом;
    \item Удаление ребер, экранируемых другими телами; 
    \item Удаление линий пересечения тел, экранируемых самими телами и другими телами, связанными отношением протыкания.
\end{itemize}

Преимуществом алгоритма является высокая точность вычислений, но он позволяет работать только с выпуклыми телами и имеет ограничение вычислительной сложности, которая возрастает пропорционально квадрату числа объектов.

\subsection{Алгоритм Варнока}

Алгоритм Варнока работает в пространстве изображения и позволяет определить, какие грани или части граней объектов сцены видимы, а какие заслонены другими объектами. 
Алгоритм прелагает разбиение области изображения нa более мелкие окна, и для каждого такого окна определяются связанные с ней многоугольники.
Те, видимость которых можно определить, изображаются нa сцене.

\includeimage
    {varnok}
    {f}
    {!ht}
    {0.50\textwidth}
    {Пример разбиения в алгоритме Варнока}


Алгоритм работает рекурсивно, что является его главным недостатком.


\subsection{Алгоритм, использующий Z-буфер}

Алгоритм, основанный на использовании Z-буфера, является простым и широко используемым инструментом. 

В данном алгоритме используется два буфера: буфер кадра и Z-буфер. 
Буфер кадра используется для заполнения атрибутов (интенсивности) каждого пикселя в пространстве изображения.  
Z-буфер – отдельный буфер глубины, используемый для запоминания координаты $z$ или глубины каждого видимого пикселя 
в пространстве изображения~\cite{ROB}.

В начале работы значения Z-буфера устанавливаются на минимальные, а в буфере кадра размещаются пиксели, описывающие задний план. 
В процессе работы каждый новый пиксель сравнивается со значениями в Z-буфере. 
Если новый пиксель ближе к наблюдателю, чем предыдущий, он заносится в буфер кадра и происходит корректировка Z-буфера. 

Основным преимуществом работы алгоритма является простота реализации. 
Трудоемкость алгоритма увеличивается линейно в зависимости от количества объектов на сцене. 
Требуется большой объем памяти.

\subsection{Алгоритм обратной трассировки лучей}

Алгоритм имеет такое название, потому что эффективнее с точки зрения вычислений отслеживать пути лучей в обратном направлении, то есть от наблюдателя к объекту. Наблюдатель видит объект посредством испускаемого источником света, который падает на этот объект и согласно законам оптики некоторым путем доходит до глаза наблюдателя. 

\includeimage
    {ray}
    {f}
    {!ht}
    {0.70\textwidth}
    {Пример работы обратной трассировки лучей}

Преимуществами являются возможность использования алгоритма в параллельных вычислительных системах и высокая реалистичность получаемого изображения, но необходимо большое количество вычислений.

\subsection{Выбор алгоритма удаления невидимых поверхностей}
Поверхность жидкости аппроксимируется треугольниками. 
Для визуализации поверхности жидкости за основу был взят 
алгоритм построчного сканирования, использующий Z-буфер. 

\newpage 

\section{Анализ алгоритмов закраски}
Методы закраски используются для затенения полигонов модели в условиях некоторой сцены,
имеющей источник освещения. 
Учитывая взаимное положение рассматриваемого полигона и источника света, можно найти уровень освещенности. 

\subsection{Простая закраска}
Вся грань закрашивается одним уровнем интенсивности, который высчитывается по закону Ламберта. 
При данной закраске все плоскости будут закрашены однотонно.

\includeimage
    {simple}
    {f}
    {!ht}
    {0.23\textwidth}
    {Пример простой закраски}

\subsection{Закраска по Гуро}
Метод Гуро устраняет дискретность изменения интенсивности и создает
иллюзию гладкой криволинейной поверхности. Он основан на интерполяции интенсивности. 

\includeimage
    {guro}
    {f}
    {!ht}
    {0.25\textwidth}
    {Пример закраски по Гуро}

\subsection{Закраска по Фонгу}
Закраска Фонга по своей идее похожа на закраску Гуро, но ее отличие состоит в том, 
что в методе Гуро по всем точкам полигона интерполируется значения интенсивностей, 
а в методе Фонга - вектора нормалей, и с их помощью для каждой точки находится значение интенсивности.

\includeimage
    {fong}
    {f}
    {!ht}
    {0.25\textwidth}
    {Пример закраски по Фонгу}

\subsection{Выбор алгоритма закраски}
Алгоритм закраски Фонга дает наиболее реалистичное изображение, в частности зеркальных бликов. 
В курсовом проекте будет использоваться метод закраски Фонга и метод закраски Гуро.

\newpage

\section{Анализ моделей освещения}
Выделяют две основные модели освещения: модель Ламберта и модель Фонга. 

\subsection{Модель освещения Ламберта}
Модель Ламберта моделирует идеальное диффузное освещение.
Освещенность в точке определяется только плотностью света в точке поверхности, а она линейно зависит от косинуса угла падения. 
При этом положение наблюдателя не имеет значения, так как диффузно отраженный свет рассеивается равномерно по всем направлениям 
Модель Ламберта является одной из самых простых моделей освещения. 

Рассеянная составляющая рассчитывается по закону косинусов (закон Ламберта)~\cite{LIGHT}.

\includeimage
    {light_lambert}
    {f}
    {!ht}
    {0.35\textwidth}
    {Модель освещения Ламберта}

Все векторы берутся единичными. Тогда косинус угла между ними совпадает 
со скалярным произведением. Формула расчета интенсивности имеет следующий вид: 

\begin{equation}
    I = I_{0} \cdot k_{d} \cdot \cos(\vec{L}, \vec{N}) \cdot I_{d} = I_{0} \cdot K_{d} \cdot (\vec{L}, \vec{N}) \cdot I_{d}
\end{equation}

Где $I$ — результирующая интенсивность света в точке; 
$I_{0}$ — интенсивность источника; 
$k_{d}$ — коэффициент диффузного освещения;  
$\vec{L}$ — вектор от точки до источника; 
$\vec{N}$— вектор нормали в точке; 
$I_{d}$ — мощность рассеянного освещения.


\subsection{Модель освещения Фонга}
Модель представляет собой комбинацию диффузной составляющей и зеркальной составляющей. 
Работает таким образом, что кроме равномерного освещения на материале может также появиться блик. 
Отраженная составляющая в точке зависит от того, насколько близки направления от рассматриваемой 
точки на точку взгляда и отраженного луча. 
Местонахождение блика на объекте, освещенном по модели Фонга, определяется из закона равенства углов падения и отражения. 
Если наблюдатель находится вблизи углов отражения, яркость соответствующей точки повышается~\cite{LIGHT}.

\includeimage
    {light_fong}
    {f}
    {!ht}
    {0.35\textwidth}
    {Модель освещения Фонга}

Для модели Фонга освещение в точке вычисляется по следующей формуле:

\begin{equation}
    I = I_{a} + I_{d} + I_{s}
\end{equation}

Где $I$ — результирующая интенсивность света в точке; 
$I_{a}$ — фоновая составляющая; 
$I_{d}$ — рассеянная составляющая;
$I_{s}$ — зеркальная составляющая;


Падающий и отраженный лучи лежат в одной плоскости с нормалью к отражающей поверхности в точке падения, и эта нормаль делит угол между лучами на две равные части. 
Таким образом отраженная составляющая освещенности в точке зависит от того, насколько близки направления на наблюдателя и отраженного луча. 
Это можно выразить следующей формулой~\cite{LIGHT}.

Формула для расчета интенсивности для модели Фонга имеет вид:
\begin{equation}
    I = K_a \cdot I_a + I_0 \cdot K_d \cdot (\vec{L}, \vec{N}) \cdot I_d + I_0 \cdot K_s \cdot (\vec{R}, \vec{V})^\alpha \cdot I_s
\end{equation}

Где $I$ — результирующая интенсивность света в точке; 
$K_a$ — коэффициент фонового освещения; 
$I_a$— интенсивность фонового освещения; 
$I_0$ — интенсивность источника; 
$K_d$ — коэффициент диффузного освещения; 
$\vec{L}$ — вектор от точки до источника; 
$\vec{N}$ — вектор нормали в точке; 
$I_d$ — интенсивность диффузного освещения; 
$K_s$ — коэффициент зеркального освещения; 
$\vec{R}$ — вектор отраженного луча; 
$\vec{V}$ — вектор от точки до наблюдателя; 
$\alpha$ — коэффициент блеска; 
$I_s$ — интенсивность зеркального освещения.


\subsection{Выбор модели освещения}
Наилучшим решением для поставленной задачи будет остановться на модели Фонга при использовании закраски по Фонгу.
Модель освещения Ламберта будет использоваться вместе с закраской по Гуро. 

\section*{Вывод}
В данном разделе проведен анализ существующих алгоритмов построения изображений и выбор подходящих алгоритмов для решения задачи.
