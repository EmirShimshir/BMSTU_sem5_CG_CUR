\chapter*{Введение}
\addcontentsline{toc}{chapter}{ВВЕДЕНИЕ}

Компьютерная графика -- это совокупность методов и способов 
преобразования информации в графическое представление при помощи ЭВМ. 
Ее применение находит широкий спектр приложений, от создания удивительных визуальных эффектов в 
компьютерных играх до моделирования сложных трехмерных объектов. 
Одной из интересных задач в области компьютерной графики является визуализация и 
моделирование волн на поверхности жидкости, особенно круговых волн, вызванных падением капли.

Цель данного курсового проекта является разработка программного обеспечения, позволяющего моделировать 
генерацию волн на поверхности жидкости.

Чтобы достигнуть поставленой цели, требуется решить следующие задачи.

\begin{enumerate}[label=---]
    \item произвести анализ существующих алгоритмов компьютерной графики;
    \item выбрать наиболее подходящие алгоритмы для решения задачи;
    \item выбрать средства реализации программного обеспечения;
    \item разработать программное обеспечение и реализовать выбранные алгоритмы и структуры данных;
    \item провести замеры временных характеристик разработанного программного обеспечения.
\end{enumerate}
