\chapter{Исследовательская часть}

\section{Постановка исследования}

Целью исследования является определение времени пересчета полигонов поверхности жидкости и времени закраски поверхности жидкости по методам Гуро и Фонга от количества полигонов.



\section{Технические характеристики}

Технические характеристики устройства, на котором выполнялось тестирование:
\begin{itemize}[label={---}]
	\item операционная система MacOS Ventura 13.5.2;
	\item 16 ГБ оперативной памяти;
    \item процессор 2,6 ГГц 6‑ядерный Intel Core i7~\cite{intel}.

\end{itemize}

Во время тестирования устройство было подключено к сети электропитания, нагружено приложениями окружения и самой системой тестирования.

\section{Время выполнения алгоритмов}

Алгоритмы тестировались при помощи функции $process\_time()$ из библиотеки $time$~\cite{pythonlangtime} языка $Python$~\cite{python}. 
Данная функция возвращает текущее значение системного процессорного времени в секундах.

Замеры времени для каждого количества полигонов проводились 20 раз. 
В качестве результата взято среднее время работы алгоритма. 

В таблице~\ref{tbl:time} приведены результаты замера времени пересчета полигонов поверхности жидкости (в микросекундах). 


\begin{table}[H]
\caption{\label{tbl:time}Результаты измерений времени пересчета полигонов поверхности жидкости}
\centering
\begin{tabular}{|r|r|}
\hline
Количество полигонов, штук & Время, микросекунды \\ \hline
8    &    213 \\ \hline
32   &    785 \\ \hline
72   &  1 753 \\ \hline
128  &  3 112 \\ \hline
200  &  4 843 \\ \hline
288  &  6 988 \\ \hline
392  &  9 506 \\ \hline
512  &  12 387 \\ \hline
648  &  15 489 \\ \hline
800  &  18 840 \\ \hline
968  &  22 818 \\ \hline
1 152 &  26 544 \\ \hline
1 352 &  30 861 \\ \hline
1 568 &  35 217 \\ \hline
\end{tabular}
\end{table}

По таблице \ref{tbl:time} был построен график на рисунке \ref{img:graph_wave}

\img{90mm}{graph_wave}{График зависимости измерений времени пересчета полигонов поверхности жидкости от количества полигонов}

\newpage

Как видно из графика, время пересчета полигонов поверхности жидкости линейно зависит от количества полигонов.

В таблице~\ref{tbl:time_draw} приведены результаты замера времени закраски поверхности жидкости по методам Гуро и Фонга (в микросекундах). 

\begin{table}[H]
\caption{\label{tbl:time_draw}Результаты измерений времени закраски поверхности жидкости по методам Гуро и Фонга}
\centering
\begin{tabular}{|c|r|r|}
\hline
\multirow{2}{*}{\shortstack{Количество полигонов, \\ штук}} & \multicolumn{2}{c|}{Время, микросекунды} \\
\cline{2-3} 
	& \multicolumn{1}{c|}{\begin{tabular}[c]{@{}c@{}}Гуро\end{tabular}} & \multicolumn{1}{c|}{\begin{tabular}[c]{@{}c@{}}Фонг\end{tabular}} \\
\hline
8 & 134 711 & 920 098 \\ \hline
32 & 158 717 & 963 961 \\ \hline
72 & 184 150 & 1 011 580 \\ \hline
128 & 206 442 & 1 072 216 \\ \hline
200 & 250 857 & 1 135 146 \\ \hline
288 & 300 670 & 1 245 661 \\ \hline
392 & 369 820 & 1 356 393 \\ \hline
512 & 440 931 & 1 477 853 \\ \hline
648 & 523 032 & 1 574 341 \\ \hline
800 & 604 238 & 1 722 877 \\ \hline
968 & 704 506 & 1 871 794 \\ \hline
1 152 & 812 945 & 2 009 417 \\ \hline
1 352 & 935 350 & 2 162 179 \\ \hline
1 568 & 1 054 775 & 2 308 075 \\ \hline
\end{tabular}
\end{table}

По таблице \ref{tbl:time_draw} был построен график на рисунке \ref{img:graph_draw}

\newpage 

\img{90mm}{graph_draw}{График зависимости измерений времени закраски поверхности жидкости по методам Гуро и Фонга от количества полигонов}

Как видно из графика, время закраски поверхности жидкости по методам Гуро и Фонга линейно зависит от количества полигонов, при этом метод закраски по Фонгу более чем в 2 раза медленне закраски по методу Гуро.
Это обусловлено тем, что интенствность каждой точки полигона в методе Фонга рассчитывается отдельно по вектору номали и вектору направления источника освещения в данной точке полигона.
В закраске по методу Гуро данная операция производится только в вершинах полгона, далее полученные значения интенствности интерполируются.

	
\section*{Вывод}

В данном разделе были исследованы результаты измерения времени пересчета полигонов поверхности жидкости и времени закраски поверхности жидкости по методам Гуро и Фонга от количества полигонов.
